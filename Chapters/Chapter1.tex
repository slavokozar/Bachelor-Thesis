% Chapter Template

\chapter{Introduction} % Main chapter title

\label{Chapter1} % Change X to a consecutive number; for referencing this chapter elsewhere, use \ref{ChapterX}


Automation, controlling and regulation are the most interesting topics for me. 
Always only in amateur and hobby level, because of rigidity and inflexibility of industrial tools and programming languages. 
While in the other sectors of IT terms like a object oriented access and code re-usability are matter of course, in industry theese are just topics of future. 

When i came across this theme topic about industrial standards of close future it realized it can be very rewarding. 
This topic is about integration of two standards which goal is to delegate ideas of commerce IT sector to industry. 


%-----------------------------------
%	Industry 4.0
%-----------------------------------
\section{Industry 4.0}
Nowadays we are standing in the time, when the fourth industrial revolutions starts.Every one of these revolutions were caused by technological improvements. First one was caused by change from labor work to mechanization. Second one was started by electrification, in this revolution electric machines were used instead of steam based motors. Third revolution was the last one, and was caused by digitization and invention of logical circuits. When we realize how much did the computers evolved it's logical that also industry has to pass another revolution.
Upcoming revolution is caused by introducting Internet of Things into industry. \cite{brettel2014virtualization}



Term Industry 4.0 was first used at the Hanover Fair in 2011, and nowadays is currently prevalent in almost every industry-related fair, conference, or call for public-funded projects. \cite{Drath2014}
This term describes project of German government aim of which is creating intelligent factories using interconnecting of manufacturing systems into Internet of Things. 


%----------------------------------------------------------------------------------------
%	Aim of thesis
%----------------------------------------------------------------------------------------

\section{Aim of thesis}
Aim of this thesis is to integrate OPC UA communication protocol into system 4DIAC - controlling framework based on IEC 61499 standard. 

Controll system created in 4DIAC framework is composed of function blocks, my task is implement communication stack inside of function blocks. Including client and also server. 

OPC UA protocol allows user to create topologically ordered web of data. My task is also to create data topology on server based on a structure of control system based on a 4DIAC framework. By integration of these two technologies I create a system in which all elements of distributed control system could load structure and status of every other element using OPC UA protocol. 



%----------------------------------------------------------------------------------------
%	Chapters overview
%----------------------------------------------------------------------------------------

\section{Chapters overview}

In following second chapter my aim is to introduce you a IEC 61499 standard and 4DIAC framework based on this standard.
I am going to show basic principles of this framework as creating application, function blocks, deploying applications.
Important part of using 4DIAC framework is compiling of your own version of 4DIAC runtime environment dedicated for your device. To this topic is dedicated whole section. Another section of this chapeter will be dedicated to compiling and runnig 4DIAC runtime on raspberry pi. 

Third chapter is dedicated to communication protocol OPC UA, ways of using this protovol and its information model. I am going to mention stacks based on OPC UA protocol. I am focusing on OPEN 62541 stack, which i have choosen to use in this thesis. 

In fourth chapter I am explaining my solution of problem explained in the previous sections of this chapter. Also I am describing example application to work with OPC UA in 4DIAC.




